\documentclass[12pt,a4paper]{scrartcl}
\usepackage[utf8]{inputenc}
\usepackage{siunitx}
\usepackage{graphicx}
\usepackage{amsmath}
\usepackage{float}

\newcommand{\n}[1]{\overline{#1}}
\newcommand{\g}[1]{\\\stackrel{\text{#1}}{=}}

\begin{document}
	\title{Grundlagen der Rechnerarchitektur\\ Übungsblatt 5}
	\date{}
	\author{Tarik Enderes, Jonas Strauch}
	\maketitle
	
	\begin{description}
		\item[1.] 
		\begin{description}
			\item[a)] Eine Schaltalgebra ist ein Spezialfall einer booleschen Algebra, in der nur 0 und 1 als Werte vorkommen.
			\item[b)] 
			\begin{description}
				\item[i.]  NOR:\\
			\begin{math}
			x_1\cdot x_2\g{Doppelnegation}\n{\n{x_1\cdot x_2}}
			\g{DeMorgan}\n{\n{x_1}+\n{x_2}}
			\g{Idempotenz}\n{\n{x_1 + x_1} + \n{x_2 + x_2}}
			\g{Definition}(x_1\n{+}x_1)\n{+}(x_2\n{+}x_2)
			\end{math}
			\\\\NAND:\\
			\begin{math}
			x_1\cdot x_2\g{Doppelnegation}\n{\n{x_1\cdot x_2}}
			\g{Idempotenz}\n{\n{x_1\cdot x_2}\cdot \n{x_1\cdot x_2}}
			\g{Definition}(x_1\n{\cdot }x_2)\n{\cdot }(x_1\n{\cdot }x_2)
			\end{math}
			\item[ii.] NOR:\\
			\begin{math}
			x_1\cdot \n{x_2}+\n{x_1}\cdot x_2
			\g{Idempotenz}x_1\cdot \n{x_2+x_2}+\n{x_1+x_1}\cdot x_2
			\g{Definition} x_1\cdot (x_2\n{+}x_2)+(x_1\n{+}x_1)\cdot x_2
			\g{i.}(x_1\n{+}x_1)\n{+}((x_2\n{+}x_2)\n{+}(x_2\n{+}x_2))
			  +((x_1\n{+}x_1)\n{+}(x_1\n{+}x_1))\n{+}(x_2\n{+}x_2)
			\g{Doppelnegation}\overline{\overline{(x_1\n{+}x_1)\n{+}((x_2\n{+}x_2)\n{+}(x_2\n{+}x_2))
				+((x_1\n{+}x_1)\n{+}(x_1\n{+}x_1))\n{+}(x_2\n{+}x_2)}}
			\g{Idempotenz}((x_1\n{+}x_1)\n{+}((x_2\n{+}x_2)\n{+}(x_2\n{+}x_2))
			\n{+}((x_1\n{+}x_1)\n{+}(x_1\n{+}x_1))\n{+}(x_2\n{+}x_2))\\\n{+}((x_1\n{+}x_1)\n{+}((x_2\n{+}x_2)\n{+}(x_2\n{+}x_2))
			\n{+}((x_1\n{+}x_1)\n{+}(x_1\n{+}x_1))\n{+}(x_2\n{+}x_2))
			\end{math}
			\\\\NAND:\\
			\begin{math}
			x_1\cdot \n{x_2}+\n{x_1}\cdot x_2
			\g{Doppelnegation}\n{\n{x_1\cdot \n{x_2}}}+\n{\n{\n{x_1}\cdot x_2}}
			\g{DeMorgan}\n{\n{x_1\cdot \n{x_2}}\cdot \n{\n{x_1}\cdot x_2}}
			\g{Idempotenz}\n{\n{x_1\cdot \n{x_2\cdot x_2}}\cdot \n{\n{x_1\cdot x_1}\cdot x_2}}
			\g{Definition}(x_1\n{\cdot }(x_2\n{\cdot }x_2))\n{\cdot }((x_1\n{\cdot }x_1)\n{\cdot }x_2)
			\end{math}
			\end{description}
		\end{description}
	\item[2.]
	\begin{description}
		\item[a)] 
		\begin{tabular}{c | c | c | c | c}
			$x_1$ & $x_2$ & $x_3$ & $x_4$ & $f(x)$\\\hline
			0 & 0 & 0 & 0 & 0 \\
			1 & 0 & 0 & 0 & 1 \\
			0 & 1 & 0 & 0 & 1 \\
			1 & 1 & 0 & 0 & 0 \\
			0 & 0 & 1 & 0 & 1 \\
			1 & 0 & 1 & 0 & 0 \\
			0 & 1 & 1 & 0 & 1 \\
			1 & 1 & 1 & 0 & 0 \\
			0 & 0 & 0 & 1 & 1 \\
			1 & 0 & 0 & 1 & 0 \\
			0 & 1 & 0 & 1 & 1 \\
			1 & 1 & 0 & 1 & 1 \\
			0 & 0 & 1 & 1 & 1 \\
			1 & 0 & 1 & 1 & 1 \\
			0 & 1 & 1 & 1 & 1 \\
			1 & 1 & 1 & 1 & 0 \\
			
		\end{tabular}
	\item[b)]
	DKNF:\\
	\begin{math}
	f(x) = x_1\n{x_2}\n{x_3}\n{x_4} + x_1\n{x_2}x_3x_4 + x_1x_2\n{x_3}x_4 + x_1\n{x_2}\n{x_3}x_4 + x_1x_2x_3\n{x_4} + x_1\n{x_2}x_3\n{x_4} + \n{x_1}\n{x_2}\n{x_3}x_4 + x_1x_2\n{x_3}\n{x_4} + \n{x_1}x_2\n{x_3}\n{x_4} + x_1\n{x_2}\n{x_3}\n{x_4} 
	\end{math}
	\\\\KKNF:\\
	\begin{math}
	f(x) = (x_1+x_2+x_3+x_4)\cdot (\n{x_1}+\n{x_2}+x_3+x_4)\cdot (\n{x_1}+x_2+\n{x_3}+x_4)\cdot (\n{x_1}+\n{x_2}+\n{x_3}+x_4)\cdot (\n{x_1}+x_2+x_3+\n{x_4})\cdot (\n{x_1}+\n{x_2}+\n{x_3}+\n{x_4})
	\end{math}
	\item[c)] 
	\begin{tabular}{c | c | c | c | c | c}
		$ & \n{x_1} & x_1 & x_1 & \n{x_1} & \\\hline
		\n{x_2} & 0 & 1 & 0 & 1 & \n{x_4}\\\hline 
		x_2 & 1 & 0 & 0 & 1 & \n{x_4}\\\hline
		x_2 & 1 & 1 & 0 & 1 & x_4\\\hline
		\n{x_2} & 1 & 0 & 1 & 1 & x_4\\\hline   
		 & \n{x_3} & \n{x_3} & x_3 & x_3 & \\$
	\end{tabular}
	\\\implies $f(x) = \n{x_1}x_3 + x_2\n{x_3}x_4 + \n{x_1}x_2\n{x_3} + \n{x_1}\n{x_3}x_4 + x_1\n{x_2}\n{x_4}\n{x_3} + \n{x_2}x_3x_4$
	\\Da die DKNF und die KKNF äquivalent sind, ist die minimale Funktion von beiden Formen Identisch.
	\end{description}
	\item[3.] 
	\begin{description}
		\item[a)] DKNF:\\ 
		\begin{math}
		f(x) = \n{x_1}\n{x_2}x_3 + \n{x_1}x_2\n{x_3} + \n{x_1}x_2x_3 + x_1\n{x_2}\n{x_3} + x_1x_2x_3
		\end{math}
		\\\\KKNF:\\
		\begin{math}
		f(x) = (x_1+x_2+x_3)\cdot (\n{x_1}+x_2+\n{x_3})\cdot(\n{x_1}+\n{x_2}+x_3) 
		\end{math}
		\item[b)] Beide Funktionen haben dieselbe Wertetabelle, aus der sie auch abgeleitet wurden. Sie liefern für dieselbe Eingabe dieselbe Ausgabe. Damit sind sie der Definition nach Äquivalent. 
		\item[c)]
		\begin{tabular}{c | c | c | c | c}
			$ & \n{x_1} & x_1 & x_1 & \n{x_1} \\\hline
			\n{x_2} & 0 & 1 & 0 & 1 \\\hline 
			x_2 & 1 & 0 & 1 & 1 \\\hline
			& \n{x_3} & \n{x_3} & x_3 & x_3 \\$
		\end{tabular}
	\\\implies $f(x) = x_1\n{x_2}\n{x_3} + \n{x_1}x_3 + x_2x_3 + \n{x_1}x_2$
	\item[d)]
	DKNF:\\ 
	\begin{math}
	g(x) = \n{x_1}\n{x_2}x_3 + \n{x_1}x_2\n{x_3} + x_1\n{x_2}\n{x_3} + x_1\n{x_2}x_3 x_1x_2\n{x_3} + x_1x_2x_3
	\end{math}
	\\\\KKNF:\\
	\begin{math}
	g(x) = (x_1+x_2+x_3)\cdot (x_1+\n{x_2}+\n{x_3})
	\end{math}
	\item[e)] 
	\begin{math}
	g(x) = (x_1+x_2+x_3)\cdot (x_1+\n{x_2}+\n{x_3}) 
	\g{Distributivität}x_1x_1 + x_1\n{x_2} + x_1\n{x_3} + x_2x_1 + x_2\n{x_2} + x_2\n{x_3} + x_3x_1 + x_3\n{x_2} + x_3\n{x_3}
	\g{Idempotenz} x_1 + x_1\n{x_2} + x_1\n{x_3} + x_2x_1 + x_2\n{x_2} + x_2\n{x_3} + x_3x_1 + x_3\n{x_2} + x_3\n{x_3}
	\g{Komplementarität} x_1 + x_1\n{x_2} + x_1\n{x_3} + x_2x_1 + x_2\n{x_3} + x_3x_1 + x_3\n{x_2}
	\g{Absorption} x_1 + x_2\n{x_3} + x_3\n{x_2}
	\end{math}
	\end{description}
	\item[4.] 
	\begin{description}
		\item[a)] 
		\begin{math}
		f(x_1, x_2, x_3) = \n{x_1}x_3 + x_2\n{x_3}
		\\\stackrel{Entwicklung nach x_1}{=}
		\n{x_1}x_3 + \n{x_1}x_2\n{x_3} + x_1x_2\n{x_3}
		\\\stackrel{Entwicklung nach x_2}{=}
		\n{x_1}\n{x_2}x_3 + \n{x_1}x_2x_3 + \n{x_1}x_2\n{x_3} + x_1x_2\n{x_3}
		\\\stackrel{Entwicklung nach x_3}{=}
		\n{x_1}\n{x_2}x_3 + \n{x_1}x_2x_3 + \n{x_1}x_2\n{x_3} + x_1x_2\n{x_3}
		\end{math}
		\item[b)] 
		\begin{math}
		f(x_1, x_2, x_3) = \n{x_1} + \n{x_2}x_3
		\\\stackrel{Entwicklung nach x_1}{=}
		\n{x_1} + \n{x_1}\n{x_2}x_3 + x_1\n{x_2}x_3
		\\\stackrel{Entwicklung nach x_2}{=}
		\n{x_1}\n{x_2} + \n{x_1}\n{x_2}x_3 + x_1\n{x_2}x_3 + \n{x_1}x_2
		\\\stackrel{Entwicklung nach x_3}{=}
		\n{x_1}\n{x_2}\n{x_3} + \n{x_1}\n{x_2}x_3 + x_1\n{x_2}x_3 + \n{x_1}x_2\n{x_3} + \n{x_1}\n{x_2}x_3 + \n{x_1}x_2x_3
		\end{math}
	\end{description}
	\end{description}
	
\end{document}